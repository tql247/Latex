
\newpage
\chapter{CÁC MÔ HÌNH TIME SERIES FORECASTING}
\centerline{\textbf{}}


 
\section{Autoregressive Integrated \\ Moving Average (ARIMA)}

\changefontsizes{13pt}
Để thực hiện, xây dựng các mô hình chuỗi thời gian, hiện tại có hai ngôn ngữ hỗ trợ tốt cho việc này là R (R có các packages như forecast và lmtest) và python. Mặc dù python dễ sử dụng và có cộng đồng lớn mạnh hơn nhiều so với R, nhưng R lại hỗ trợ tốt hơn python trong thống kê và hiện thực hóa mô hình chuỗi thời gian. Đó cũng là một trong những lý do mà các nhà thống kê và kinh tế lượng ưa chuộng sử dụng R hơn. Tuy nhiên, ở bài báo cáo này chúng ta chỉ dừng lại với python trong việc tìm hiểu về cách xây dựng mô hình ARIMA.




\section{Generalize Autoregressive Conditionally \\ Hestoroscedastic (GARCH)}

\vspace{1cm}
\changefontsizes{13pt}
GARCH (Generalize Autoregressive Conditionally Hestoroscedastic) là một dạng mô hình thuộc nhóm hồi qui chuỗi thời gian thường được áp dụng phổ biến trong các dự báo kinh tế và tài chính. Ban đầu, mô hình gốc ARCH được giới thiệu bởi Engle (1982) để hồi qui các chuỗi dừng nhưng có mối quan hệ phi tuyến tính. Nguyên nhân của sự phi tuyến tính này xuất phát từ việc phương sai sai số của mô hình thay đổi dẫn đến việc áp dụng các mô hình dựa trên giả định phương sai sai số không đổi như ARIMA hoặc ARMA không đạt kết quả chuẩn xác. Trong khi đó hiện tượng phương sai sai số thay đổi khá phổ biến trong các chuỗi chứng khoán và tài chính bởi luôn có những cú sốc kinh tế thường đến bất ngờ và là một nhân tố mà mô hình không thể giải thích được. Trong biểu diễn phân phối của chuỗi chúng ta sẽ thấy chuỗi có dạng đuôi béo (fat-tail). Điều đó cho thấy chuỗi có nhiều outlier tập trung ở các khoảng cận trên hoặc cận dưới.

Bên cạnh đó một mô hình ARCH là một mô hình được xây dựng với biến mục tiêu là phương sai của một chuỗi số. Do đó ARCH thường được sử dụng để giải thích sự thay đổi hoặc sự biến động của phương sai chuỗi. Mặc dù ARCH có thể được sử dụng để mô tả sự tăng tiến của phương sai qua thời gian nhưng thay vào đó nó thường được sử dụng để mô tả sự gia tăng biến động chuỗi trong một khoảng thời gian ngắn.

Trong kinh tế và tài chính mô hình ARCH thường được áp dụng với những chuỗi tăng dần (hoặc giảm dần) theo chu kì như giá chứng khoán, GDP, qui mô dân số,…. Do đó chúng ta cần các biến để diễn tả được xu thế tăng (giảm) này nhưng vẫn đảm bảo được chuỗi dừng như là điều kiện cần của chuỗi thời gian. Vì thế các biến thường được sử dụng trong mô hình GARCH thường là:

Phần trăm lãi / lỗ qua thời gian:
% yt=(xt−xt−1)xt−1
%Lợi suất danh mục thông qua sai phân bậc 1 Logarit: yt=log(xt)−log(xt−1)
Ngoài ra mô hình GARCH có thể sử dụng bất kì biến nào khác có yếu tố tăng giảm phương sai theo chu kì. Chẳng hạn như số dư từ chính mô hình ARIMA đã được hồi qui.



\section{Long Short Term Memories (LSTM)}







