\chapter{Tổng kết}

%
\bigskip


\setlength{\parindent}{0.0cm}
\textbf{\textbf{Kết quả}}

\vspace{0.5cm}
\setlength{\parindent}{0cm}
\textbf{Kết quả đạt được}

\vspace{0.5cm}


\setlength{\parindent}{0cm}
%fill here
Qua những mội dung được trình bày ở trên, ta có thể thấy bài báo cáo đã trình bày một cách cơ bản và đầy đủ nhất chủ đề tìm hiểu về time series forecasting. Qua đó ta có thể hiểu được thế nào là time series, ứng dụng và cách triển khai thuật toán. Nó mang nhiều giá trị hiện thực không chỉ trong nghiên cứu khoa học mà còn có thể áp dụng vào thực tiễn của cuộc sống.


\vspace{0.5cm}
\textbf{Hạn chế}


\vspace{0.5cm}


Bài báo cáo tuy đã trình bày tốt nhất có thể nhưng vẫn còn tồn tại một vài thiếu xót nhỏ. Các phần cần nêu theo mục đích của đề tài đã được hoàn thành nhưng chỉ ở mức là tổng quát, chung nhất, chưa đi vào chi tiết và trình bày cụ thể.


\vspace{0.5cm}

\setlength{\parindent}{0.0cm}
\textbf{Hướng phát triển}



\vspace{0.5cm}


Có nhiều hướng phát triển khác nhau cho các bài time series forecasting. Ngoài việc xây dựng thuật toán theo các phương pháp trên thì còn có nhiều phương pháp khác, được thiết kế lại để tối ưu hóa hiệu năng, độ chính xác,... cho những trường hợp và mục đích riêng biệt. Ngành công nghệ thông tin ngày càng phát triển, đặc biệt là các lĩnh vực trí tuệ nhân tạo, học máy, học sâu. Nên trong tương lai, chắc chắn sẽ có thêm nhiều cách tiếp cận mới, hiệu quả và chính xác hơn. Không chỉ hữu ích trong bài toán time series forecasting mà còn nhiều bài toán khác nữa.

