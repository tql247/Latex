\chapter{Tổng kết}

%
\bigskip


\setlength{\parindent}{0.0cm}
\textbf{\textbf{Kết quả}}

\vspace{0.5cm}
\setlength{\parindent}{0cm}
\textbf{Kết quả đạt được}

\vspace{0.5cm}


\setlength{\parindent}{0cm}
%fill here
Như vậy chúng ta đã đi qua hai phần của bài báo cáo time series forecasting trong deep learning. Hy vọng qua hai bài báo cáo này thì người đọc có thể nắm chắc được nền tảng kiến thức căn bản của mô hình time series và ứng dụng nó vào thực tế, nhằm phục vụ cho các nghiên cứu của cá nhân hoặc phục vụ mục đích tìm hiểu của bản thân.

\vspace{0.5cm}
\textbf{Hạn chế}


\vspace{0.5cm}


Bài báo cáo tuy đã trình bài chi tiết về các mô hình (ARIMA và LSTM), tuy nhiên vẫn còn một vài mô hình dự báo đáng chú ý khác vẫn chưa được đề cập trong bài báo cáo này. Người đọc có thể tìm hiểu thêm: RNN, ResNet, ode\_gru\_bayes,...


\vspace{0.5cm}

\setlength{\parindent}{0.0cm}
\textbf{Hướng phát triển}



\vspace{0.5cm}


Từ những kiến thức đã có, hy vọng người đọc có thể tự xây dựng một mô hình phù hợp với nhu cầu và mục đích sử dụng của riêng mình. Hoặc có thể nghiên cứ, đề ra một phát kiến mới đóng góp cho sự phát triển chung của ngành AI nói chung và lĩnh vực time series forecasting nói riêng.

