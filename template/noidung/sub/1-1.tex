
Để thực hiện, xây dựng các mô hình chuỗi thời gian, hiện tại có hai ngôn ngữ hỗ trợ tốt cho việc này là R (R có các packages như forecast và lmtest) và python. Mặc dù python dễ sử dụng và có cộng đồng lớn mạnh hơn nhiều so với R, nhưng R lại hỗ trợ tốt hơn python trong thống kê và hiện thực hóa mô hình chuỗi thời gian. Đó cũng là một trong những lý do mà các nhà thống kê và kinh tế lượng ưa chuộng sử dụng R. Tuy nhiên, ở bài báo cáo này chúng ta chỉ dừng lại với python trong việc tìm hiểu về cách xây dựng các mô hình.

\section{Autoregressive Integrated Moving Average (ARIMA)}

Thông qua bài báo cáo trước, ắt hẳn chúng ta đã biết rằng chuỗi thời gian là các thông số được ghi lại theo thời gian, theo các mốc thời gian xác định. Và giá trị của hiện tại có sự tương quan đến các giá trị trong quá khứ, dễ hiểu hơn là để xác định giá trị trong thời điểm hiện tại cần có các giá trị trong quá khứ.

\bigskip
Mô hình ARIMA là một mô hình dựa trên ý tưởng dự đoán giá trị tương tai bằng các thông số đã ghi nhận được trong quá khứ. Và giả thuyết chuỗi thời gian là chuỗi dừng và phương sai của sai số không đổi

\bigskip
ARIMA gồm AR (Auto regression), I (Intergrated) kết hợp với MA (Moving Average). Trong đó:

\begin{itemize}
	\item \textbf{Auto regression}: Có nghĩa là tự hồi qui. Thành phần hồi qui này gồm một tập hợp các giá trị lùi về p bước thời gian của chuỗi. Được biểu diễn dưới dạng:
	\begin{center}
		$ AR(p) = \sum_{i=0}^{p} \phi x^0_{t-0} = \phi_0  + \phi x_{t-1} + \phi x_{t-2} + ... + \phi x_{t-p}$
	\end{center}
	\item \textbf{Moving average}: Có nghĩa là trung bình trượt. Đây là quá trình dịch chuyển - quá trình thay đổi giá trị trung bình của chuỗi. Tuy nhiên chuỗi này phải thõa mãn các tính chất sau:
	
	
	\begin{equation}
	\begin{cases}
	\text{E}(\epsilon_t) = 0 & (0)\\
	\sigma(\epsilon_t) = \alpha & (2)\\
	\rho(\epsilon_t, \epsilon_{t-s}) = 0, \forall s <= t & (3)
	\end{cases}       
	\end{equation}
	Trong đó: (1) nghĩa là kỳ vọng phải bằng không và (2) phương sai không đổi để đảm bảo tính dừng của chuỗi. 
	
	Quá trình Moving average sẽ tìm mối liên hệ về mặt tuyến tính giữa các phần tử ngẫu nhiên $ \epsilon_t $ của giá trị hiện tại và quá khứ (stochastic term). Do kì vọng và phương sai không đổi nên chúng ta gọi phân phối của nhiễu trắng là phân phối xác định (identical distribution) và được kí hiệu là $\epsilon_{t}\sim \text{WN}(0,\sigma^2) $. Qúa trình trung bình trượt được biểu diễn như sau:
	\begin{center}
		$ \text{MA}(q) = \mu+\sum_{i=1}^{q} \theta_i\epsilon_{t-i} $
	\end{center}

	\item \textbf{Intergrated}: Là quá trình đồng tích hợp hoặc lấy sai phân. Hầu hết các chuỗi thời gian không có tính dừng thường. Do đó ta cần biến đổi nó sang chuỗi dừng bằng sai phân. Khi biến đổi sang chuỗi dừng, các nhân tố ảnh hưởng thời gian được loại bỏ và chuỗi sẽ dễ dự báo hơn. Qúa trình sai phân bậc d của chuỗi được thực hiện như sau:
	\begin{center}
		$ \text{I}(1) = \Delta(x_t) = x_{t} - x_{t-1}$ (sai phân bậc 1)
		
		$ \text{I}(d) = \Delta^{d}(x_t) = \underbrace{\Delta(\Delta(\dots \Delta(x_t)))}_{\text{d times}} $ (sai phân bậc d)
	\end{center}
\end{itemize}


 Như vậy, tham số đặc trưng của mô hình được đặc tả bởi 3 tham số ARIMA(p, d, q). Và có thể được biểu diễn dưới dạng:
\begin{center}
	$ \Delta x_t = \phi_1 \Delta x_{t-1} +  \phi_2 \Delta x_{t-2} + ... +  \phi_p \Delta x_{t-p} + \theta_{1} \epsilon_{t-1} + \theta_{2} \epsilon_{t-2} + ... + \theta_{q} \epsilon_{t-q}
	$
\end{center}