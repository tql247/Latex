\newpage
\changefontsizes{16pt}
\centerline{\textbf{\hyperlink{page.7}{Chương 4: KẾT LUẬN}}}


%
\bigskip
\changefontsizes{14pt}

\setlength{\parindent}{0.0cm}
\textbf{4.1 Kết quả đạt được.}

\smallskip
\changefontsizes{14pt}
\setlength{\parindent}{0.5cm}
\textbf{4.1.1 Kết quả.}

\smallskip
\changefontsizes{13pt}

\leftskip0.5cm

\setlength{\parindent}{1cm}
%fill here
Qua những điều được nêu lên ở các chương phía trên, ta có thể thấy bài báo cáo đã trình bày một cách cơ bản nhất chủ đề của đồ án là quy hoạch động, thông qua ba bài toán được giao. Mỗi bài toán đều được nêu lên nội dung, chỉ ra cách tiếp cận chung cho bài toán dạng quy hoạch động và cũng nói lên được ý nghĩa khoa học của nó không chỉ trong nghiên cứu mà còn có thể áp dụng vào những việc thực tiễn của cuộc sống.

Bài báo cáo còn nêu lên phương hướng giải quyết một cách tổng quát nhất, đặc trưng của phương hướng đó, đồng thời là các cách đánh giá gồm nhiều tiêu chí, để từ nó ta có thể thấy độ hiễu quả của giải thuật được nêu.

Phần thực nghiệm cũng cho thấy những kết quả đáng mong đợi từ việc áp dụng những phương pháp đã nêu vào giải quyết các vấn đề được giao. Đó việc đưa ra được mã giả kết hợp cùng việc giải thích, code demo đã cho ra nhiều kết quả theo đúng kì vọng mà những testcase của người lập trình đã đề ra. Qua đó, ta có thể kết luận một lần nữa là bài báo cáo cơ bản đã hoàn thành yêu cầu đề ra.

\leftskip0cm
\bigskip
\changefontsizes{14pt}

\setlength{\parindent}{0.5cm}
\textbf{4.1.2 Hạn chế.}

\smallskip
\changefontsizes{13pt}

\leftskip0.5cm

\setlength{\parindent}{1cm}

Bài báo cáo tuy đã hoàn thiện cơ bản nhất có thể nhưng vẫn còn tồn tại một vài thiếu xót nhỏ. Các phần theo yêu cầu đã được hoàn thành nhưng còn ở mức là tổng quát, chung nhất, chưa đi vào một cách cụ thể, có thể do phần quy hoạch động còn ít tài liệu tham khảo. Phương pháp giải quyết cũng chính là đề tài được nêu nên rất khó cho việc sáng tạo hay đổi mới. Vì vậy, bài báo cáo nếu muốn nâng cao hơn cần tìm hiểu rất nhiều mới có thể cải tiến và phát triển hơn được.

\leftskip0cm
%fill here

\bigskip
\changefontsizes{14pt}

\setlength{\parindent}{0.0cm}
\textbf{4.2 Hướng phát triển.}

\smallskip
\changefontsizes{13pt}

\setlength{\parindent}{1cm}
Có nhiều hướng phát triển khác nhau cho các bài toán dạng quy hoạch động. Ngoài việc xây dựng thuật toán theo kiểu bài báo cáo đã nêu thì các còn có nhiều thuật toán tối ưu khác giúp cho việc nâng cao hơn trng ứng dụng sau này. Ngành công nghệ thông tin ngày càng phát triển, bên trong đó là các lĩnh vực trí tuệ nhân tạo, học máy cũng ứng dụng việc sử dụng quy hoạch động để tạo nên cách thuật toán tìm kiếm, hay build một con AI chatbox,… và còn nhiều ứng dụng khác nữa. Có thể xây dựng cho các chương trình một giao diện người dùng để có thể dễ tương tác hơn cho người dùng cuối.

