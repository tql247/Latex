\newpage
\changefontsizes{16pt}
\centerline{\textbf{\hyperlink{page.6}{CHƯƠNG 2: GIẢI QUYẾT BÀI TOÁN}}}



\bigskip
\changefontsizes{14pt}

\setlength{\parindent}{0.0cm}
\textbf{2.1 Tổng quát về phương pháp giải quyết bài toán.}

\smallskip
\changefontsizes{13pt}
\setlength{\parindent}{1.0cm}

%fill here
Các bài toán sẽ được chia nhỏ ra thành các phần đơn giản, mà có thể dễ dàng giải quyết bằng các phương pháp thông thường. Ta dùng đệ quy hoặc không đệ quy kết hợp các biến lưu trữ để chia nhỏ bài toán.


\bigskip
\changefontsizes{14pt}

\setlength{\parindent}{0.0cm}
\textbf{2.2 Đặc trưng của giải pháp đề xuất.}


\smallskip
\changefontsizes{13pt}
\setlength{\parindent}{1.0cm}

%fill here
Phương pháp giải quyết đã đề xuất ở trên mang tính tổng quát, áp dụng được cho hầu hết trường hợp. Dễ thực hiện, số câu lệnh ngắn.

\bigskip
\changefontsizes{14pt}

\setlength{\parindent}{0.0cm}
\textbf{2.3 Cách đánh giá.}

\smallskip
\changefontsizes{13pt}
\setlength{\parindent}{1cm}
Các tiêu chí để đánh giá gồm có:

\leftskip1cm
- Tính chính xác của kết quả trả về.

- Độ phức tạp O của thuật toán.

- Mức độ khó dễ của việc cài đặt giải thuật.
